In this chapter we will discuss the execution instructions of the code. The project was developed in \href{https://docs.python.org/3/whatsnew/3.9.html}{Python3.9} however, it's compatible with python3.6-3.9. The requirements can be found in the \path{./requirements.txt}. The code has been compiled to be able to run it as an executable without installing the dependancies with \href{https://www.pyinstaller.org/}{pyinstaller}.

This instructions have been tested for UNIX OS. This has not been tested for Windows or WSL. We provide two alternatives to run the project: 

\section{Data retrieval}

We provide a bash script to automatically download the data and add headers to the csv file, as it's useful for our usecase. The bas script is located in \path{./getData.sh} and can be run as: 

\begin{lstlisting}[language=bash]
./getData.sh
\end{lstlisting}

The contents of the script are as follows:

\lstinputlisting[]{../../getData.sh}

The getData script will download the dataset requiered, add the headers and do a split if the dataset is not yet in the dataset.

\section{Binary execution instructions}

The binary files are located in \path{./dist/*} folders. Two scripts can be run with the following commands and will scan the \path{data/} folder for files with extensions \path{.train} and \path{.test} respectively and run the train and test executables in \path{./dist/train/train} and \path{./dist/test/test} respectively.

\begin{lstlisting}[language=bash]
./train.sh
./test.sh
\end{lstlisting}

The contents of the ./train.sh file are as follows:

\lstinputlisting[]{../../train.sh}

The contents of the ./test.sh file are as follows:

\lstinputlisting[]{../../test.sh}

The scripts will generate log files in the \path{log/} folder and the models and rules in the \path{models/} folder.

\section{Python environment}

If not running in linux, follow this steps to run the project:

\begin{itemize}
    \item from the PW1 folder
    \item create conda/virtualenv and activate
    \begin{lstlisting}[language=bash]
# anaconda
conda create --name sel3.9 python=3.9
conda activate sel3.9
pip install -r requirements.txt

# virtualenv
python3 -m venv venv/
source venv/bin/activate
pip install -r requirements.txt
    \end{lstlisting}
    \item download data from:
    \begin{enumerate}
        \item https://archive.ics.uci.edu/ml/machine-learning-databases/hayes-roth/hayes-roth.data
        \item https://archive.ics.uci.edu/ml/machine-learning-databases/hayes-roth/hayes-roth.test
        \item https://archive.ics.uci.edu/ml/machine-learning-databases/chess/king-rook-vs-king-pawn/kr-vs-kp.data
        \item https://archive.ics.uci.edu/ml/machine-learning-databases/car/car.data
    \end{enumerate}
    \item filter unused column from hayes-roth and rename hayes-roth.data to hayes-roth.train with the following python command:
    \begin{lstlisting}[language=bash]
python -c "import pandas as pd; data=pd.read_csv('data/hayes-roth.data'); data=data.drop('name', axis=1); data.to_csv('data/hayes-roth.train', index=False)"
    \end{lstlisting}
    then remove the hayes-roth.data file.
    \item create train and test splits for the data files:
    \begin{lstlisting}[language=bash]
python scripts/split_train_test.py -i data/car.data
python scripts/split_train_test.py -i data/kr-vs-kp.data
    \end{lstlisting}
\end{itemize}

\section{Python Execution instructions}

To train a model run the following command from PW1:

\begin{lstlisting}[language=bash]
python src/train.py -i data/car.train -f models/ -l log/car.train.log
python src/train.py -i data/hayes-roth.train -f models/ -l log/hayes-roth.train.log
python src/train.py -i data/kr-vs-kp.train -f models/ -l log/kr-vs-kp.train.log
\end{lstlisting}

And for inference run:
\begin{lstlisting}[language=bash]
python src/test.py -i data/car.test -r models/car.rules
python src/test.py -i data/hayes-roth.test -r models/hayes-roth.rules
python src/test.py -i data/kr-vs-kp.test -r models/kr-vs-kp.rules
\end{lstlisting}

\section{Compilation instructions}

To compile the project, a bash script \path{./build.sh} is provided and it is imperative to have pyinstaller in the python environment.

\begin{lstlisting}[language=bash]
./build.sh
\end{lstlisting}

\lstinputlisting[]{../../build.sh}
